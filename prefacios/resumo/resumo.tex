\chapter*{Resumo}

Devido à necessidade de softwares com alta qualidade e pouco tempo para atividades de validação e verificação, técnicas de teste de software foram desenvolvidas com a finalidade de identificar e corrigir erros pertinentes ao sistema, dando suporte ao desenvolvimento e integrando todo o ciclo de vida da aplicação. 

Dentre as técnicas criadas, uma delas é a automação de testes, que em meio à correria do desenvolvimento de software se torna uma poderosa estratégia para minimizar o esforço com atividades de testes, principalmente os de regressão, que são: diminuição de tempo gasto em atividades de testes, reuso de testes, maior cobertura dos testes aplicados entre outros. 

Com o objetivo de encontrar falhas o mais cedo possível, este trabalho apresenta uma estratégia de automação de testes durante todo o ciclo de vida do desenvolvimento de uma aplicação.Participando de todas as fases do desenvolvimento, realizando atividades em conjunto com o desenvolvedor, para cobrir todos os níveis possíveis de implementação do sistema. Os resultados apurados destacam o impacto de tal estratégia na realização dos testes, viabiliza a aplicação de testes automatizados durante o desenvolvimento e ressalta a importância que a qualdiade final pode melhorar com este tipo de estratégia. 



\textbf{Palavras-chave:} teste de software, automação de testes, teste de regressão.
